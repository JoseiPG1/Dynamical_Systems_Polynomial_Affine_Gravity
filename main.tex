\documentclass{article}
\usepackage[utf8]{inputenc}
\usepackage{amsmath}
\usepackage{amsfonts}
\usepackage{cite} 
\usepackage{breqn}
\usepackage{graphicx}
\usepackage{hyperref}

\allowdisplaybreaks

%Useful definitions
\providecommand{\Rie}[3]{\mathcal{R}_{#1}{}^{ #2}{}_{#3}}
\providecommand{\ctG}[3]{\Gamma_{#1}{}^{ #2}{}_{#3}}
\providecommand{\ctg}[3]{\gamma_{#1}{}^{ #2}{}_{#3}}
\providecommand{\B}[3]{\mathcal{B}_{#1}{}^{ #2}{}_{#3}}
\providecommand{\P}[3]{\mathcal{P}_{#1}{}^{ #2}{}_{#3}}
\providecommand{\A}[1]{\mathcal{A}_{#1}}
\providecommand{\Ri}[1]{\mathcal{R}_{#1}}
\providecommand{\deV}[1]{\mathrm{d}V^{#1}}


\title{Torsion effects in Polynomial Affine Gravity as a dynamical system}
\author{Jos\'e Perdiguero Garate}

\begin{document}

\maketitle

\section{Introduction}

\section{The model}


\subsection{The formulation}

The most general action (up to topological invariants and boundary terms) in four dimensions is given by
\begin{equation}
    \label{PAG_action}
\begin{split}
S & =
    \int  \mathrm{d}V^{\alpha \beta \gamma \delta} \bigg[
    B_1 \mathcal{R}_{\mu\nu}{}^{\mu}{}_{\rho}\mathcal{B}_{\alpha}{}^{\nu}{}_{\beta}\mathcal{B}_{\gamma}{}^{\rho}{}_{\delta}
    + B_2 \mathcal{R}_{\alpha\beta}{}^{\mu}{}_{\rho} \mathcal{B}_{\gamma}{}^{\nu}{}_{\delta} \mathcal{B}_{\mu}{}^{\rho}{}_{\nu}
    + B_3 \mathcal{R}_{\mu\nu}{}^{\mu}{}_{\alpha} \mathcal{B}_{\beta}{}^{\nu}{}_{\gamma} \mathcal{A}_\delta
    \\
    & \quad
    + B_4 \mathcal{R}_{\alpha\beta}{}^{\sigma}{}_{\rho}\mathcal{B}_{\gamma}{}^{\rho}{}_{\delta}\mathcal{A}_\sigma
    + B_5 \mathcal{R}_{\alpha \beta}{}^{\rho}{}_{\rho} \mathcal{B}_{\gamma}{}^{\sigma}{}_{\delta} \mathcal{A}_\sigma
    + C_1 \mathcal{R}_{\mu\alpha}{}^{\mu}{}_{\nu} \nabla_\beta \mathcal{B}_{\gamma}{}^{\nu}{}_{\delta}
    \\
    & \quad
    + C_2 \mathcal{R}_{\alpha\beta}{}^{\rho}{}_{\rho} \nabla_\sigma \mathcal{B}_{\gamma}{}^{\sigma}{}_{\delta}
    + D_1 \mathcal{B}_{\nu}{}^{\mu}{}_{\lambda} \mathcal{B}_{\mu}{}^{\nu}{}_{\alpha} \nabla_\beta \mathcal{R}_{\gamma}{}^{\lambda}{}_{\delta}
    + D_2 \mathcal{B}_{\alpha}{}^{\mu}{}_{\beta} \mathcal{B}_{\mu}{}^{\lambda}{}_{\nu} \nabla_{\lambda} \mathcal{B}_{\gamma}{}^{\nu}{}_{\delta}
    \\
    & \quad
    + D_3 \mathcal{B}_{\alpha}{}^{\mu}{}_{\nu}\mathcal{B}_{\beta}{}^{\lambda}{}_{\gamma} \nabla_\lambda \mathcal{B}_{\mu}{}^{\nu}{}_{\delta}
    + D_4 \mathcal{B}_{\alpha}{}^{\lambda}{}_{\beta}\mathcal{B}_{\gamma}{}^{\sigma}{}_{\delta}\nabla_\lambda \mathcal{A}_\sigma
    + D_5 \mathcal{B}_{\alpha}{}^{\lambda}{}_{\beta} \mathcal{A}_\sigma \nabla_\lambda \mathcal{B}_{\gamma}{}^{\sigma}{}_{\delta}
    \\
    &\quad
    + D_6 \mathcal{B}_{\alpha}{}^{\lambda}{}_{\beta}\mathcal{A}_\gamma \nabla_\lambda A_\delta
    + D_7\mathcal{B}_{\alpha}{}^{\lambda}{}_{\beta} \mathcal{A}_\lambda \nabla_\gamma A_\delta
    + E_1\nabla_\rho \mathcal{B}_{\alpha}{}^{\rho}{}_{\beta} \nabla_\sigma \mathcal{B}_{\gamma}{}^{\sigma}{}_{\delta}
    \\
    &\quad
    + E_2 \nabla_\rho \mathcal{B}_{\alpha}{}^{\rho}{}_{\beta} \nabla_\gamma \mathcal{A}_\delta
    + F_1 \mathcal{B}_{\alpha}{}^{\mu}{}_{\beta} \mathcal{B}_{\gamma}{}^{\sigma}{}_{\delta} \mathcal{B}_{\mu}{}^{\lambda}{}_{\rho} \mathcal{B}_{\sigma}{}^{\rho}{}_{\lambda}
    + F_2\mathcal{B}_{\alpha}{}^{\mu}{}_{\beta} \mathcal{B}_{\gamma}{}^{\nu}{}_{\lambda} \mathcal{B}_{\delta}{}^{\lambda}{}_{\rho} \mathcal{B}_{\mu}{}^{\rho}{}_{\nu}
    \\
    &\quad
    + F_3 \mathcal{B}_{\nu}{}^{\mu}{}_{\lambda} \mathcal{B}_{\mu}{}^{\nu}{}_{\alpha} \mathcal{B}_{\beta}{}^{\lambda}{}_{\gamma} \mathcal{A}_\delta
    + F_4 \mathcal{B}_{\alpha}{}^{\mu}{}_{\beta}\mathcal{B}_{\gamma}{}^{\nu}{}_{\delta}\mathcal{A}_\mu \mathcal{A}_\nu \bigg].
\end{split}
\end{equation}


\begin{align}
    \label{G_ansatz}
    \ctG{t}{t}{t} & =f(t) & \ctG{i}{t}{j} & = g(t) S_{i j} \\
    \ctG{t}{i}{j} &= h(t) \delta^{i}_{j} & \ctG{i}{j}{k} & = \ctg{i}{j}{k},
\end{align}
where $S_{ij}$ is the three-dimensional rank two symmetric tensor defined as follow
\begin{equation}
    S_{i j}=\left(\begin{array}{ccc}
    \frac{1}{1-\kappa r^2} & 0 & 0 \\
    0 & r^2 & 0 \\
    0 & 0 & r^2 \sin ^2 \theta
    \end{array}\right),
\end{equation}
and $\gamma$ is the symmetric connection compatible with desired symmetries written as
\begin{align}
    \ctg{r}{r}{r} & = \frac{\kappa r}{1 - \kappa r^2} & \ctg{\theta}{r}{\theta} & = \kappa r^3 - r &
    \ctg{\varphi}{r}{\varphi} & = \left(\kappa r^3 - r\right)\sin^2\theta & \ctg{r}{\theta}{\theta} & = \frac{1}{r} \\
    \ctg{\varphi}{\theta}{\varphi} & = -\cos\theta\sin\theta & \ctg{r}{\varphi}{\varphi} & = \frac{1}{r} & 
    \ctg{\theta}{\varphi}{\varphi} & = \frac{\cos \theta}{\sin \theta}.
\end{align}

Interestingly, the affine function $f(t)$ can be set equal to zero, under a reparametrisation of the time coordinate, for more information
on this type of transformation, please refer to Ref. \cite{Castillo_Felisola_2022_EPJC}. Therefore, there are only two non trivial functions
to define completely the symmetric part of the connection.

\begin{equation}
\label{B_ansatz}
\begin{aligned}
    \B{\theta}{r}{\varphi} & = \psi (t) r^2\sin\theta \sqrt{1 - \kappa r^2} &
    \B{r}{\theta}{\varphi} & =\frac{\psi (t) \sin \theta}{\sqrt{1 - \kappa r^2}} & 
    \B{r}{\varphi}{\theta} & =\frac{\psi(t)}{ \sqrt{1-\kappa r^{2}} \sin \theta}.
\end{aligned}
\end{equation}
Notice the trace-less part of the torsion tensor has only one time-dependent function to defined the tensor completely.
\begin{equation}
    \label{A_ansatz}
    \A{t} = \eta(t).
\end{equation}
There is only one time-dependent function to define completely the vectorial part of the torsion tensor.

Finally, the complete set of field equations are obtained through Kijowski's formalism, see Refs.[XX] for each irreducible 
field of the affine connection. The explicit form of the field equations can be found in Ref.\cite{Castillo-Felisola_2023}.


\section{Dynamical system technique}
\label{sec: dynamical_system}

\begin{dmath}
    \label{Feq_1}
    \left(B_3\left(\dot{g} + gh + 2\kappa\right) - 2B_4\left(\dot{g} - gh\right) + 2D_6\eta g - 2F_3\psi^2\right)\psi = 0,
\end{dmath}
\begin{dmath}
    \label{Feq_2}
    \left(B_3\eta\psi -2B_4\eta\psi + C_1\left(\dot{\psi} - 2h\psi\right)\right)g = 0,
\end{dmath}
\begin{dmath}
    \label{Feq_3}
    \left(B_3 + 2B_4\right)\eta g\psi + 2C_1\left(\kappa\psi + 4gh\psi - g\dot{\psi} - \psi\dot{g}\right) + 2\psi^3\left(2D_2 - D_1 - D_3\right) = 0,
\end{dmath}
\begin{dmath}
    \label{Feq_4}
    B_3\left(\eta\left(h\psi - \dot{\psi}\right) -\psi\dot{\eta}\right) - 2B_4\left(\eta\left(-h\psi - \dot{\psi}\right) -\psi\dot{\eta}\right) 
    + C_1\left(4h^2\psi + 2\psi\dot{h} -\ddot{\psi}\right) + D_6\eta^2\psi = 0,
\end{dmath}
\begin{dmath}
    \label{Feq_5}
    B_3\left(\dot{g} + gh + 2\kappa\right)\eta - 2B_4\left(\dot{g} - gh\right)\eta + C_1\left(2\kappa h + 4gh^2 + 2g\dot{h} - \ddot{g}\right) +
    6h\psi^2\left(2D_2 - D_1 - D_3\right) + D_6 \eta^2 g - 6F_3\eta\psi^2 = 0
\end{dmath}

From this, we solve eq.\eqref{Feq_B1_2} to find an expression for $\eta(t)$
\begin{equation}
    \label{B1_eta}
    \eta(t) = \left(\frac{2h\psi - \dot{\psi}}{\psi}\right)\left(\frac{C_1}{B_3 - 2B_4}\right),
\end{equation}
replacing the above expression for $\eta(t)$ into eq. \eqref{Feq_B1_4}, leads to two sub-branches for the $h(t)$ function
\begin{align}
    \label{B1_h}
    h(t) & = \frac{\dot{\psi}}{2\psi} & h(t) & = \frac{\dot{\psi}}{\psi}\left(\frac{C_1 D_6}{3B_3^2 - 8B_3B_4 + B_4^2 + 2C_1D_6}\right)
\end{align}
using the simplest form of $h(t)$ function, then eq. \eqref{Feq_B1_3} leads to
\begin{equation}
    -2\left(D_1 - 2D_2 + D_3\right)\psi^3 + 2C_1\left(\psi\left(\kappa - \dot{g}\right) + g\dot{\psi}\right) = 0,
\end{equation}
which is a first order differential equation which can be solve for $g(t)$ in terms of the $\psi$ function
\begin{equation}
    \label{B1_g}
    g(t) = \psi(t) \left(g_0 + \int_1^t \left(\frac{\kappa}{\psi(\tau)} - \psi(\tau) \left(\frac{D_1 - 2D_2 + D_3}{C_1}\right)\right) \mathrm{d}\tau\right)
\end{equation}
the above solution also solves eq. \eqref{Feq_B1_5}, where $g_0$ is an integration constant. Then, eq. \eqref{Feq_B1_1} becomes 
an integro-differential equation of first order
\begin{dmath}
    \dot{\psi}\left(g_0 + \int_1^t \left(\frac{\kappa}{\psi(\tau)} - \psi(\tau) \left(\frac{D_1 - 2D_2 + D_3}{C_1}\right)\right) \mathrm{d}\tau\right)\left(\frac{3B_3 - 2B_4}{2}\right) -
    \psi^2 \frac{\left(B_3 - 2B_4\right)\left(D_1 - 2D_2 + D_3\right) + 2C_1 F_3}{C_1} + \kappa \left(3B_3 - 2B_4\right) = 0.
\end{dmath}
The above equation can be solved for the special case $\kappa = 0$, with the variable change $\psi (t) = \dot{\phi}(t)$, then
\begin{dmath}
    \ddot{\phi}\left(g_0  - \phi\alpha\right)\beta - \dot{\phi}^2 \gamma  = 0,
\end{dmath}
where
\begin{align}
    \alpha & = \left(\frac{D_1 - 2D_2 + D_3}{C_1}\right) & \beta & = \left(\frac{3B_3 - 2B_4}{2}\right) \\
    \gamma & = \frac{\left(B_3 - 2B_4\right)\left(D_1 - 2D_2 + D_3\right) + 2C_1 F_3}{C_1},
\end{align}
whose solution
\begin{equation}
    \phi(t) = \frac{g_0}{\alpha} + \frac{\left(\phi_0 \left(\alpha\beta + \gamma\right) \left(t +  \phi_1\right)\right)^{\frac{\alpha\beta}{\alpha\beta + \gamma}}}{\alpha\beta},
\end{equation}
where $\phi_0$ and $\phi_1$ are integration constant. From this
\begin{equation}
    \psi(t) = \phi_0 \left(\phi_0 \left(\alpha\beta + \gamma\right)\left(t + \phi_1\right)\right)^{-\frac{\gamma}{\alpha\beta + \gamma}}.
\end{equation}
From the above expression and using the relations in eqs. \eqref{B1_eta}, \eqref{B1_h} and \eqref{B1_g} it is straightforward to find the rest of the
affine functions
\begin{align}
    \eta(t) & = 0,\\
    h(t) & = - \frac{\gamma}{2\left(\alpha\beta + \gamma\right)\left(t + \phi_1\right)},\\
    g(t) & = \phi_0 \left(\left(\alpha\beta + \gamma\right)\phi_0 \left(t + \phi_1\right)\right)^{-\frac{\gamma}{\alpha\beta + \gamma}}
    \left(g_0 - \frac{\left(\left(\alpha\beta + \gamma\right)\phi_0\left(t + \phi_1\right)\right)^{\frac{\alpha\beta}{\alpha\beta + \gamma}}}{\beta}\right)
\end{align}

\section{Final remarks}
\label{sec: final_remarks}

\bibliographystyle{unsrt}
\bibliography{References}

\end{document}

