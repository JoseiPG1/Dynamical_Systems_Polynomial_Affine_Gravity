\documentclass{article}
\usepackage[utf8]{inputenc}
\usepackage{amsmath}
\usepackage{amsfonts}
\usepackage{cite} 
\usepackage{breqn}
\usepackage{graphicx}
\usepackage{hyperref}

\allowdisplaybreaks

%Useful definitions
\providecommand{\Rie}[3]{\mathcal{R}_{#1}{}^{ #2}{}_{#3}}
\providecommand{\ctG}[3]{\Gamma_{#1}{}^{ #2}{}_{#3}}
\providecommand{\ctg}[3]{\gamma_{#1}{}^{ #2}{}_{#3}}
\providecommand{\B}[3]{\mathcal{B}_{#1}{}^{ #2}{}_{#3}}
\providecommand{\P}[3]{\mathcal{P}_{#1}{}^{ #2}{}_{#3}}
\providecommand{\A}[1]{\mathcal{A}_{#1}}
\providecommand{\Ri}[1]{\mathcal{R}_{#1}}
\providecommand{\deV}[1]{\mathrm{d}V^{#1}}


\title{Torsion effects in Polynomial Affine Gravity as a dynamical system}
\author{Jos\'e Perdiguero Garate}

\begin{document}

\maketitle

\section{Introduction}

\section{The model}




\subsection{The formulation}
The most general action (up to topological invariant and boundary terms) in four dimensions is given by
\begin{equation}
    \label{PAG_action}
\begin{split}
S & =
    \int  \mathrm{d}V^{\alpha \beta \gamma \delta} \bigg[
    B_1 \mathcal{R}_{\mu\nu}{}^{\mu}{}_{\rho}\mathcal{B}_{\alpha}{}^{\nu}{}_{\beta}\mathcal{B}_{\gamma}{}^{\rho}{}_{\delta}
    + B_2 \mathcal{R}_{\alpha\beta}{}^{\mu}{}_{\rho} \mathcal{B}_{\gamma}{}^{\nu}{}_{\delta} \mathcal{B}_{\mu}{}^{\rho}{}_{\nu}
    + B_3 \mathcal{R}_{\mu\nu}{}^{\mu}{}_{\alpha} \mathcal{B}_{\beta}{}^{\nu}{}_{\gamma} \mathcal{A}_\delta
    \\
    & \quad
    + B_4 \mathcal{R}_{\alpha\beta}{}^{\sigma}{}_{\rho}\mathcal{B}_{\gamma}{}^{\rho}{}_{\delta}\mathcal{A}_\sigma
    + B_5 \mathcal{R}_{\alpha \beta}{}^{\rho}{}_{\rho} \mathcal{B}_{\gamma}{}^{\sigma}{}_{\delta} \mathcal{A}_\sigma
    + C_1 \mathcal{R}_{\mu\alpha}{}^{\mu}{}_{\nu} \nabla_\beta \mathcal{B}_{\gamma}{}^{\nu}{}_{\delta}
    \\
    & \quad
    + C_2 \mathcal{R}_{\alpha\beta}{}^{\rho}{}_{\rho} \nabla_\sigma \mathcal{B}_{\gamma}{}^{\sigma}{}_{\delta}
    + D_1 \mathcal{B}_{\nu}{}^{\mu}{}_{\lambda} \mathcal{B}_{\mu}{}^{\nu}{}_{\alpha} \nabla_\beta \mathcal{R}_{\gamma}{}^{\lambda}{}_{\delta}
    + D_2 \mathcal{B}_{\alpha}{}^{\mu}{}_{\beta} \mathcal{B}_{\mu}{}^{\lambda}{}_{\nu} \nabla_{\lambda} \mathcal{B}_{\gamma}{}^{\nu}{}_{\delta}
    \\
    & \quad
    + D_3 \mathcal{B}_{\alpha}{}^{\mu}{}_{\nu}\mathcal{B}_{\beta}{}^{\lambda}{}_{\gamma} \nabla_\lambda \mathcal{B}_{\mu}{}^{\nu}{}_{\delta}
    + D_4 \mathcal{B}_{\alpha}{}^{\lambda}{}_{\beta}\mathcal{B}_{\gamma}{}^{\sigma}{}_{\delta}\nabla_\lambda \mathcal{A}_\sigma
    + D_5 \mathcal{B}_{\alpha}{}^{\lambda}{}_{\beta} \mathcal{A}_\sigma \nabla_\lambda \mathcal{B}_{\gamma}{}^{\sigma}{}_{\delta}
    \\
    &\quad
    + D_6 \mathcal{B}_{\alpha}{}^{\lambda}{}_{\beta}\mathcal{A}_\gamma \nabla_\lambda A_\delta
    + D_7\mathcal{B}_{\alpha}{}^{\lambda}{}_{\beta} \mathcal{A}_\lambda \nabla_\gamma A_\delta
    + E_1\nabla_\rho \mathcal{B}_{\alpha}{}^{\rho}{}_{\beta} \nabla_\sigma \mathcal{B}_{\gamma}{}^{\sigma}{}_{\delta}
    \\
    &\quad
    + E_2 \nabla_\rho \mathcal{B}_{\alpha}{}^{\rho}{}_{\beta} \nabla_\gamma \mathcal{A}_\delta
    + F_1 \mathcal{B}_{\alpha}{}^{\mu}{}_{\beta} \mathcal{B}_{\gamma}{}^{\sigma}{}_{\delta} \mathcal{B}_{\mu}{}^{\lambda}{}_{\rho} \mathcal{B}_{\sigma}{}^{\rho}{}_{\lambda}
    + F_2\mathcal{B}_{\alpha}{}^{\mu}{}_{\beta} \mathcal{B}_{\gamma}{}^{\nu}{}_{\lambda} \mathcal{B}_{\delta}{}^{\lambda}{}_{\rho} \mathcal{B}_{\mu}{}^{\rho}{}_{\nu}
    \\
    &\quad
    + F_3 \mathcal{B}_{\nu}{}^{\mu}{}_{\lambda} \mathcal{B}_{\mu}{}^{\nu}{}_{\alpha} \mathcal{B}_{\beta}{}^{\lambda}{}_{\gamma} \mathcal{A}_\delta
    + F_4 \mathcal{B}_{\alpha}{}^{\mu}{}_{\beta}\mathcal{B}_{\gamma}{}^{\nu}{}_{\delta}\mathcal{A}_\mu \mathcal{A}_\nu \bigg].
\end{split}
\end{equation}

\subsection{Cosmological ansatz}

\begin{align}
    \label{G_ansatz}
    \ctG{t}{t}{t} & =f(t) & \ctG{i}{t}{j} & = g(t) S_{i j} \\
    \ctG{t}{i}{j} &= h(t) \delta^{i}_{j} & \ctG{i}{j}{k} & = \ctg{i}{j}{k},
\end{align}
where $S_{ij}$ is the three-dimensional rank two symmetric tensor defined as follow
\begin{equation}
    S_{i j}=\left(\begin{array}{ccc}
    \frac{1}{1-\kappa r^2} & 0 & 0 \\
    0 & r^2 & 0 \\
    0 & 0 & r^2 \sin ^2 \theta
    \end{array}\right),
\end{equation}
and $\gamma$ is the symmetric connection compatible with desired symmetries written as
\begin{align}
    \ctg{r}{r}{r} & = \frac{\kappa r}{1 - \kappa r^2} & \ctg{\theta}{r}{\theta} & = \kappa r^3 - r &
    \ctg{\varphi}{r}{\varphi} & = \left(\kappa r^3 - r\right)\sin^2\theta & \ctg{r}{\theta}{\theta} & = \frac{1}{r} \\
    \ctg{\varphi}{\theta}{\varphi} & = -\cos\theta\sin\theta & \ctg{r}{\varphi}{\varphi} & = \frac{1}{r} & 
    \ctg{\theta}{\varphi}{\varphi} & = \frac{\cos \theta}{\sin \theta}.
\end{align}

Interestingly, the affine function $f(t)$ can be set equal to zero, under a parametrization of the time coordinate, for more information
on this type of transformation, please refer to Ref. \cite{Castillo_Felisola_2022_EPJC}. Therefore, there are only two non trivial functions
to define completely the symmetric part of the connection.

\begin{equation}
\label{B_ansatz}
\begin{aligned}
    \B{\theta}{r}{\varphi} & = \psi (t) r^2\sin\theta \sqrt{1 - \kappa r^2} &
    \B{r}{\theta}{\varphi} & =\frac{\psi (t) \sin \theta}{\sqrt{1 - \kappa r^2}} & 
    \B{r}{\varphi}{\theta} & =\frac{\psi(t)}{ \sqrt{1-\kappa r^{2}} \sin \theta}.
\end{aligned}
\end{equation}
Notice the trace-less part of the torsion tensor has only one time-dependent function to defined the tensor completely.
\begin{equation}
    \label{A_ansatz}
    \A{t} = \eta(t).
\end{equation}
There is only one time-dependent function to define completely the vectorial part of the torsion tensor.

To summarize, the complete set of our cosmological ansatz compatible with the symmetries of the cosmological principle are the following
time dependent functions
\begin{equation}
    \begin{aligned}
        \Gamma & \to \left\{g(t), h(t)\right\} & \mathcal{B} & \to \left\{\psi(t)\right\} & \mathcal{A} & \to \left\{\eta(t)\right\}.
    \end{aligned}
\end{equation}


\section{Dynamical system technique}
\label{sec: dynamical_system}

\begin{dmath}
    \label{Feq_1}
    \left(B_3\left(\dot{g} + gh + 2\kappa\right) - 2B_4\left(\dot{g} - gh\right) + 2D_6\eta g - 2F_3\psi^2\right)\psi = 0,
\end{dmath}
\begin{dmath}
    \label{Feq_2}
    \left(B_3\eta\psi -2B_4\eta\psi + C_1\left(\dot{\psi} - 2h\psi\right)\right)g = 0,
\end{dmath}
\begin{dmath}
    \label{Feq_3}
    \left(B_3 + 2B_4\right)\eta g\psi + 2C_1\left(\kappa\psi + 4gh\psi - g\dot{\psi} - \psi\dot{g}\right) + 2\psi^3\left(2D_2 - D_1 - D_3\right) = 0,
\end{dmath}
\begin{dmath}
    \label{Feq_4}
    B_3\left(\eta\left(h\psi - \dot{\psi}\right) -\psi\dot{\eta}\right) - 2B_4\left(\eta\left(-h\psi - \dot{\psi}\right) -\psi\dot{\eta}\right) 
    + C_1\left(4h^2\psi + 2\psi\dot{h} -\ddot{\psi}\right) + D_6\eta^2\psi = 0,
\end{dmath}
\begin{dmath}
    \label{Feq_5}
    B_3\left(\dot{g} + gh + 2\kappa\right)\eta - 2B_4\left(\dot{g} - gh\right)\eta + C_1\left(2\kappa h + 4gh^2 + 2g\dot{h} - \ddot{g}\right) +
    6h\psi^2\left(2D_2 - D_1 - D_3\right) + D_6 \eta^2 g - 6F_3\eta\psi^2 = 0
\end{dmath}

From this, we solve eq.\eqref{Feq_B1_2} to find an expression for $\eta(t)$
\begin{equation}
    \label{B1_eta}
    \eta(t) = \left(\frac{2h\psi - \dot{\psi}}{\psi}\right)\left(\frac{C_1}{B_3 - 2B_4}\right),
\end{equation}
replacing the above expression for $\eta(t)$ into eq. \eqref{Feq_B1_4}, leads to two sub-branches for the $h(t)$ function
\begin{align}
    \label{B1_h}
    h(t) & = \frac{\dot{\psi}}{2\psi} & h(t) & = \frac{\dot{\psi}}{\psi}\left(\frac{C_1 D_6}{3B_3^2 - 8B_3B_4 + B_4^2 + 2C_1D_6}\right)
\end{align}
using the simplest form of $h(t)$ function, then eq. \eqref{Feq_B1_3} leads to
\begin{equation}
    -2\left(D_1 - 2D_2 + D_3\right)\psi^3 + 2C_1\left(\psi\left(\kappa - \dot{g}\right) + g\dot{\psi}\right) = 0,
\end{equation}
which is a first order differential equation which can be solve for $g(t)$ in terms of the $\psi$ function
\begin{equation}
    \label{B1_g}
    g(t) = \psi(t) \left(g_0 + \int_1^t \left(\frac{\kappa}{\psi(\tau)} - \psi(\tau) \left(\frac{D_1 - 2D_2 + D_3}{C_1}\right)\right) \mathrm{d}\tau\right)
\end{equation}
the above solution also solves eq. \eqref{Feq_B1_5}, where $g_0$ is an integration constant. Then, eq. \eqref{Feq_B1_1} becomes 
an integro-differential equation of first order
\begin{dmath}
    \dot{\psi}\left(g_0 + \int_1^t \left(\frac{\kappa}{\psi(\tau)} - \psi(\tau) \left(\frac{D_1 - 2D_2 + D_3}{C_1}\right)\right) \mathrm{d}\tau\right)\left(\frac{3B_3 - 2B_4}{2}\right) -
    \psi^2 \frac{\left(B_3 - 2B_4\right)\left(D_1 - 2D_2 + D_3\right) + 2C_1 F_3}{C_1} + \kappa \left(3B_3 - 2B_4\right) = 0.
\end{dmath}
The above equation can be solved for the special case $\kappa = 0$, with the variable change $\psi (t) = \dot{\phi}(t)$, then
\begin{dmath}
    \ddot{\phi}\left(g_0  - \phi\alpha\right)\beta - \dot{\phi}^2 \gamma  = 0,
\end{dmath}
where
\begin{align}
    \alpha & = \left(\frac{D_1 - 2D_2 + D_3}{C_1}\right) & \beta & = \left(\frac{3B_3 - 2B_4}{2}\right) \\
    \gamma & = \frac{\left(B_3 - 2B_4\right)\left(D_1 - 2D_2 + D_3\right) + 2C_1 F_3}{C_1},
\end{align}
whose solution
\begin{equation}
    \phi(t) = \frac{g_0}{\alpha} + \frac{\left(\phi_0 \left(\alpha\beta + \gamma\right) \left(t +  \phi_1\right)\right)^{\frac{\alpha\beta}{\alpha\beta + \gamma}}}{\alpha\beta},
\end{equation}
where $\phi_0$ and $\phi_1$ are integration constant. From this
\begin{equation}
    \psi(t) = \phi_0 \left(\phi_0 \left(\alpha\beta + \gamma\right)\left(t + \phi_1\right)\right)^{-\frac{\gamma}{\alpha\beta + \gamma}}.
\end{equation}
From the above expression and using the relations in eqs. \eqref{B1_eta}, \eqref{B1_h} and \eqref{B1_g} it is straightforward to find the rest of the
affine functions
\begin{align}
    \eta(t) & = 0,\\
    h(t) & = - \frac{\gamma}{2\left(\alpha\beta + \gamma\right)\left(t + \phi_1\right)},\\
    g(t) & = \phi_0 \left(\left(\alpha\beta + \gamma\right)\phi_0 \left(t + \phi_1\right)\right)^{-\frac{\gamma}{\alpha\beta + \gamma}}
    \left(g_0 - \frac{\left(\left(\alpha\beta + \gamma\right)\phi_0\left(t + \phi_1\right)\right)^{\frac{\alpha\beta}{\alpha\beta + \gamma}}}{\beta}\right)
\end{align}

\section{Final remarks}
\label{sec: final_remarks}

\appendix

\section{Appendix I: Field equation}
\label{sec:Appendix_I}

Using Kijowkski's formalism for an arbitrary tensor-field $\Phi^{\alpha_1\alpha_2 ...}{}_{\beta_1\beta_2}$, its field equations are
\begin{equation}
    \nabla_\mu \Pi_{\Phi}{}^{\mu\beta_1\beta_2 ...}{}_{\alpha_1\alpha_2 ...}  = \frac{\partial^* \mathcal{L}}{\partial \Phi^{\alpha_1\alpha_2 ...}{}_{\beta_1\beta_2}},
\end{equation}
where the asterisk on the right-hand side denotes the partial derivatives with respect to the connection that is not
contained in the Riemann curvature tensor, additionally, the canonically conjugated momenta is defined by
\begin{equation}
    \nabla_\mu \Pi_{\Phi}{}^{\mu\beta_1\beta_2 ...}{}_{\alpha_1\alpha_2 ...}
    = \frac{\partial \mathcal{L}}{\partial \left(\partial_\mu \Phi^{\alpha_1\alpha_2 ...}{}_{\beta_1\beta_2}\right)} 
    = \frac{\partial \mathcal{L}}{\partial \Phi^{\alpha_1\alpha_2 ...}{}_{\mu\beta_1\beta_2}}.
\end{equation}


\subsection{Vectorial torsion field equation}

The field equations of the vectorial part of the torsion tensor are
\begin{equation}
    \nabla_\mu \Pi_{\mathcal{A}}{}^{\mu\nu}  = \frac{\partial^* \mathcal{L}}{\partial \A{\nu}},
\end{equation}
where the asterisk on the right-hand side denotes the partial derivatives with respect to the connection that is not
contained in the Riemann curvature tensor, additionally, the canonically conjugated momenta is defined by
\begin{equation}
    \nabla_\mu \Pi_{\mathcal{A}}{}^{\mu\nu}  
    = \frac{\partial \mathcal{L}}{\partial \left(\partial_\mu \A{\nu}\right)} 
    = \frac{\partial \mathcal{L}}{\partial \A{\mu\nu}}.
\end{equation}
The field equations are:
\begin{align*}
    B_3: & -\Rie{\sigma\tau}{\sigma}{\alpha}\B{\beta}{\tau}{\gamma}\deV{\alpha\beta\gamma\nu} \\
    B_4: & -\Rie{\alpha\beta}{\nu}{\sigma}\B{\gamma}{\sigma}{\tau}\deV{\alpha\beta\gamma\tau} \\
    B_5: & -\Rie{\alpha\beta}{\rho}{\rho}\B{\gamma}{\nu}{\tau}\deV{\alpha\beta\gamma\tau} \\
    D_4: & \nabla_\mu\left(\B{\alpha}{\mu}{\beta}\B{\gamma}{\nu}{\tau}\deV{\alpha\beta\gamma\tau}\right) \\
    D_5: & -\B{\alpha}{\sigma}{\beta}\nabla_\sigma\B{\gamma}{\nu}{\tau}\deV{\alpha\beta\gamma\tau} \\
    D_6: & \nabla_\mu\left(\B{\alpha}{\mu}{\beta}\A{\gamma}\deV{\alpha\beta\gamma\nu}\right) + \B{\alpha}{\mu}{\beta}\nabla_\mu\A{\gamma}\deV{\alpha\beta\gamma\nu} \\
    D_7: & \nabla_\mu\left(\B{\alpha}{\sigma}{\beta}\A{\sigma}\deV{\alpha\beta\mu\nu}\right) - \B{\alpha}{\nu}{\beta}\mathcal{F}_{\gamma\tau}\deV{\alpha\beta\gamma\tau} \\
    E_2: & \nabla_\mu \left(\nabla_\sigma\B{\alpha}{\sigma}{\beta}\right)\deV{\alpha\beta\mu\nu}\\
    F_3: & -\B{\sigma}{\tau}{\lambda}\B{\tau}{\sigma}{\alpha}\B{\beta}{\lambda}{\gamma}\deV{\alpha\beta\gamma\nu}\\
    F_4: & -2\B{\alpha}{\sigma}{\beta}\B{\gamma}{\nu}{\tau}\A{\sigma}\deV{\alpha\beta\gamma\tau} 
\end{align*}

\subsection{Trace-less torsion field equation}

The field equations of the trace-less part of the torsion tensor are
\begin{equation}
    \nabla_\mu \Pi_{\mathcal{B}}{}^{\mu\nu}{}_{\lambda}{}^{\rho}  = \frac{\partial^* \mathcal{L}}{\partial \B{\nu}{\lambda}{\rho}},
\end{equation}
where the asterisk on the right-hand side denotes the partial derivatives with respect to the connection that is not
contained in the Riemann curvature tensor, additionally, the canonically conjugated momenta is defined by
\begin{equation}
    \nabla_\mu \Pi_{\mathcal{B}}{}^{\mu\nu}{}_{\lambda}{}^{\rho}  
    = \frac{\partial \mathcal{B}}{\partial \left(\partial_\mu \B{\nu}{\lambda}{\rho}\right)} 
    = \frac{\partial \mathcal{L}}{\partial  \B{\mu\nu}{\lambda}{\rho}}.
\end{equation}
The field equations are:
\begin{align*}
    B_1: & -4\Rie{\mu\sigma}{\mu}{\lambda}\B{\gamma}{\sigma}{\delta}\deV{\nu\rho\gamma\delta} - \frac{4}{3}\Rie{\mu\tau}{\mu}{\sigma}\B{\gamma}{\sigma}{\delta}\delta^{[\nu}_{\lambda}\deV{\rho]\tau\gamma\delta} - \frac{4}{3}\Rie{\mu\sigma}{\mu}{\tau}\B{\gamma}{\sigma}{\delta}\delta^{[\nu}_{\lambda}\deV{\rho]\tau\gamma\delta}\\
    B_2: & -2\Rie{\alpha\beta}{\mu}{\sigma}\B{\mu}{\sigma}{\lambda}\deV{\nu\rho\alpha\beta} - 2\Rie{\alpha\beta}{[\nu}{\lambda}\B{\gamma}{\rho]}{\delta}\deV{\alpha\beta\gamma\delta} - \frac{4}{3}\Rie{\alpha\beta}{\mu}{\sigma}\B{\mu}{\sigma}{\tau}\delta^{[\nu}_\lambda\deV{\rho]\tau\alpha\beta}\\
        & -\frac{2}{3}\Rie{\alpha\beta}{\tau}{\tau}\B{\gamma}{[\nu}{\delta}\delta^{\rho]}_\lambda\deV{\alpha\beta\gamma\delta} + \frac{2}{3}\Rie{\alpha\beta}{[\nu}{\tau}\delta^{\rho]}_{\lambda}\B{\gamma}{\tau}{\delta}\deV{\alpha\beta\gamma\delta}\\
    B_3: & -\frac{2}{3}\Rie{\mu\lambda}{\mu}{\alpha}\A{\beta}\deV{\nu\rho\alpha\beta} - \frac{4}{3}\Rie{\mu\tau}{\mu}{\alpha}\A{\beta}\delta^{[\nu}_{\lambda}\deV{\rho]\tau\alpha\beta}\\
    B_4: & -2\Rie{\alpha\beta}{\sigma}{\lambda} \A{\sigma}\deV{\nu\rho\alpha\beta} - \frac{4}{3}\Rie{\alpha\beta}{\sigma}{\tau}\A{\sigma}\delta^{[\nu}_{\lambda}\deV{\rho]\tau\alpha\beta}\\
    B_5: & -2\Rie{\alpha\beta}{\tau}{\tau} \A{\lambda}\deV{\nu\rho\alpha\beta} - \frac{4}{3}\Rie{\alpha\beta}{\tau}{\tau}\A{\delta}\delta^{[\nu}_{\lambda}\deV{\rho]\delta\alpha\beta}\\
    C_1: & \nabla_\mu \left(-2 \Rie{\sigma\alpha}{\sigma}{\lambda} \deV{\mu\nu\rho\alpha} + \frac{4}{3}\Rie{\sigma\alpha}{\sigma}{\tau}\delta^{[\nu}_{\lambda}\deV{\rho]\mu\tau\alpha}\right) \\
    C_2: & \nabla_\mu \left(2\Rie{\alpha\beta}{\sigma}{\sigma}\delta^{\mu}_{\lambda}\deV{\nu\rho\alpha\beta} + \frac{4}{3}\Rie{\alpha\beta}{\sigma}{\sigma}\delta^{[\nu}_{\lambda}\deV{\rho]\mu\alpha\beta}\right) \\
    D_1: & \nabla_{\mu}\left( -2 \B{\sigma}{\theta}{\lambda} \B{\theta}{\sigma}{\alpha}\deV{\mu\nu\rho\alpha} \right) - 2\B{\lambda}{[\nu}{\alpha} \nabla_{\beta} \B{\gamma}{\rho]}{\delta} \deV{\alpha\beta\gamma\delta} - 2 \B{\lambda}{[\nu\lvert}{\sigma} \nabla_{\beta} \B{\gamma}{\sigma}{\delta} \deV{\lvert\rho]\beta\gamma\delta} \\
        & -\frac{2}{3} \delta^{[\nu}_{\lambda} \B{\tau}{\rho]}{\alpha} \nabla_{\beta} \B{\gamma}{\tau}{\delta} \deV{\alpha\beta\gamma\delta} + \frac{2}{3}\B{\tau}{[\nu}{\sigma}\delta^{\rho]}_{\lambda} \nabla_{\beta} \B{\gamma}{\sigma}{\delta} \deV{\tau\beta\gamma\delta}\\
    D_2: & \nabla_{\mu} \left( 2\B{\alpha}{\sigma}{\beta} \B{\sigma}{\mu}{\lambda} \deV{\nu\rho\alpha\beta} +\frac{4}{3} \B{\alpha}{\sigma}{\beta} \B{\sigma}{\mu}{\tau} \delta^{[\nu}_{\lambda} \deV{\rho]\tau\alpha\beta} \right) - 2\B{\lambda}{\mu}{\sigma}\nabla_{\mu}\B{\alpha}{\sigma}{\beta} \deV{\nu\rho\alpha\beta} - 2\B{\alpha}{[\nu}{\beta} \nabla_{\lambda} \B{\gamma}{\rho]}{\delta} \deV{\alpha\beta\gamma\delta}\\
        & - \frac{4}{3} \B{\tau}{\mu}{\sigma} \nabla_{\mu} \B{\alpha}{\sigma}{\beta} \delta^{[\nu}_{\lambda} \deV{\rho]\tau\alpha\beta} - \frac{2}{3}\B{\alpha}{\tau}{\beta} \nabla_{\tau} \B{\gamma}{[\nu}{\delta} \delta^{\rho]}_{\lambda} \deV{\alpha\beta\gamma\delta} + \frac{2}{3} \B{\alpha}{[\nu}{\beta} \delta^{\rho]}_{\lambda} \nabla_{\tau} \B{\gamma}{\tau}{\delta}\deV{\alpha\beta\gamma\delta}\\
    D_3: &  \nabla_{\mu} \left( -2 \B{\beta}{\mu}{\gamma} \B{\alpha}{[\nu}{\lambda} \deV{\rho]\alpha\beta\gamma} -\frac{2}{3} \B{\beta}{\mu}{\gamma} \B{\alpha}{[\nu}{\tau} \delta^{\rho]}_{\lambda}\deV{\alpha\beta\gamma\tau} \right) - 2\B{\beta}{\mu}{\gamma}\nabla_{\mu}\B{\lambda}{[\rho}{\delta}\deV{\nu]\beta\gamma\delta} \\
        &  -2 \B{\gamma}{\mu}{\sigma} \nabla_{\lambda} \B{\mu}{\sigma}{\delta}\deV{\nu\rho\gamma\delta} - \frac{2}{3}\B{\beta}{\mu}{\gamma} \nabla_{\mu} \B{\tau}{[\nu}{\delta} \delta^{\rho]}_{\lambda} \deV{\tau\beta\gamma\delta} - \frac{4}{3} \B{\gamma}{\mu}{\sigma} \nabla_{\tau} \B{\mu}{\sigma}{\delta} \delta^{[\nu}_{\lambda} \deV{\rho]\tau\gamma\delta}\\
    D_4: & -4\B{\alpha}{\sigma}{\beta} \nabla_{(\lambda} \A{\sigma)} \deV{\nu\rho\alpha\beta} - \frac{4}{3} \B{\alpha}{\sigma}{\beta}\nabla_{\tau}\A{\sigma} \delta^{[\nu}_{\lambda} \deV{\rho]\tau\alpha\beta} - \frac{4}{3}\B{\alpha}{\sigma}{\beta} \nabla_{\sigma}\A{\tau} \delta^{[\nu}_{\lambda}\deV{\rho]\tau\alpha\beta}\\
    D_5: & \nabla_{\mu} \left( 2\B{\alpha}{\mu}{\beta} \A{\lambda} \deV{\nu\rho\alpha\beta} + \frac{4}{3}\B{\alpha}{\mu}{\beta} \A{\tau} \delta^{[\nu}_{\lambda} \deV{\rho]\tau\alpha\beta} \right) - 2 \nabla_{\lambda} \B{\alpha}{\sigma}{\beta}\A{\sigma}\deV{\nu\rho\alpha\beta} - \frac{4}{3}\nabla_{\tau} \B{\alpha}{\sigma}{\beta} \A{\sigma} \delta^{[\nu}_{\lambda}\deV{\rho]\tau\alpha\beta}\\
    D_6: & -2\A{\gamma}\nabla_\lambda \A{\delta}\deV{\nu\rho\gamma\delta} - \frac{4}{3}\A{\gamma}\nabla_\tau\A{\delta}\delta^{[\nu}_{\lambda}\deV{\rho]\tau\gamma\delta} \\
    D_7: & -\A{\lambda}\mathcal{F}_{\gamma\delta}\deV{\nu\rho\gamma\delta} - \frac{4}{3}\A{\tau}\mathcal{F}_{\gamma\delta} \delta^{[\nu}_{\lambda}\deV{\rho]\tau\gamma\delta}\\
    E_1: & \nabla_\mu \left( 4\delta^\mu_\lambda \nabla_\sigma \B{\alpha}{\sigma}{\beta}\deV{\nu\rho\alpha\beta} + \frac{8}{3}\nabla_\sigma \B{\alpha}{\sigma}{\beta}\delta^{[\nu}_\rho \deV{\rho]\mu\alpha\beta}\right) \\
    E_2: & \nabla_\mu \left(2\delta^\mu_\lambda \mathcal{F}_{\alpha\beta}\deV{\nu\rho\alpha\beta} + \frac{4}{3}\mathcal{F}_{\alpha\beta}\delta^{[\nu}_\lambda\deV{\rho]\mu\alpha\beta}\right)\\
    F_1: & -4\B{\alpha}{\mu}{\beta}\B{\mu}{\sigma}{\tau}\B{\lambda}{\tau}{\sigma}\deV{\nu\rho\alpha\beta} -4\B{\alpha}{\mu}{\beta}\B{\mu}{[\nu}{\delta}\B{\mu}{\rho]}{\lambda}\deV{\alpha\beta\gamma\delta}  -\frac{8}{3}\B{\alpha}{\mu}{\beta}\B{\mu}{\sigma}{\tau}\B{k}{\tau}{\sigma}\delta^{[\nu}_\lambda\deV{\rho]k\alpha\beta} \\
    F_2: & -2\B{\alpha}{\mu}{\sigma}\B{\beta}{\sigma}{\tau}\B{\lambda}{\tau}{\mu}\deV{\nu\rho\alpha\beta} + 2\B{\alpha}{\mu}{\beta}\B{\mu}{\sigma}{\lambda}\B{\gamma}{[\nu}{\sigma}\deV{\rho]\alpha\beta\gamma} - 2\B{\alpha}{\mu}{\beta}\B{\gamma}{\sigma}{\lambda}\B{\mu}{[\nu}{\sigma}\deV{\rho]\alpha\beta\gamma} \\
        & -2\B{\alpha}{[\nu}{\beta}\B{\gamma}{\rho]}{\sigma}\B{\delta}{\sigma}{\lambda}\deV{\alpha\beta\gamma\delta} -\frac{4}{3}\B{\alpha}{\mu}{\sigma}\B{\beta}{\sigma}{\tau}\B{k}{\tau}{\mu}\delta^{[\nu}_{\lambda}\deV{\rho]k\alpha\beta} - \frac{2}{3}\B{\alpha}{\mu}{\beta}\B{\mu}{\sigma}{\tau}\B{\gamma}{[\nu}{\sigma}\delta^{\rho]}_\lambda\deV{\alpha\beta\tau\gamma} \\
        & -\frac{2}{3}\B{\alpha}{\tau}{\beta}\B{\delta}{\sigma}{\tau}\B{\gamma}{[\nu}{\sigma}\delta^{\rho]}_\lambda \deV{\alpha\beta\gamma\delta}\\
    F_3: & -2\B{\lambda}{[\nu}{\alpha}\B{\beta}{\rho]}{\gamma}\A{\delta}\deV{\alpha\beta\gamma\delta} -2\B{\alpha}{\sigma}{\beta}\A{\gamma}\B{\lambda}{[\nu}{\sigma}\deV{\rho]\alpha\beta\gamma} + \frac{2}{3}\B{\beta}{\tau}{\gamma}\A{\delta}\B{\tau}{[\nu}{\alpha}\delta^{\rho]}_{\lambda}\deV{\alpha\beta\gamma\delta} \\
        & + \frac{2}{3}\B{\beta}{\tau}{\gamma}\A{\delta}\B{\tau}{[\nu}{\alpha}\delta^{\rho]}_{\lambda}\deV{\alpha\beta\gamma\delta} + \frac{2}{3}\B{\alpha}{\sigma}{\beta}\A{\gamma}\B{\tau}{[\nu}{\sigma}\delta^{\rho]}_{\lambda}\deV{\tau\alpha\beta\gamma}\\
    F_4: & -4\B{\alpha}{\mu}{\beta}\A{\mu}\A{\lambda}\deV{\nu\rho\alpha\beta} - \frac{8}{3}\B{\alpha}{\mu}{\beta}\A{\mu}\A{\tau}\delta^{[\nu}_{\lambda}\deV{\rho]\tau\alpha\beta} 
\end{align*}



\subsection{Symmetric connection field equation}

The field equations of the symmetric part of the connection are
\begin{equation}
    \nabla_\mu \Pi_{\Gamma}{}^{\mu\nu}{}_{\lambda}{}^{\rho}  = \frac{\partial^* \mathcal{L}}{\partial \Gamma_{\nu}{}^{\lambda}{}_{\rho}},
\end{equation}
where the asterisk on the right-hand side denotes the partial derivatives with respect to the connection that is not
contained in the Riemann curvature tensor, additionally, the canonically conjugated momenta is defined by
\begin{equation}
    \nabla_\mu \Pi_{\Gamma}{}^{\mu\nu}{}_{\lambda}{}^{\rho}  
    = \frac{\partial \mathcal{L}}{\partial \left(\partial_\mu \Gamma_{\nu}{}^{\lambda}{}_{\rho}\right)} 
    = \frac{\partial \mathcal{L}}{\partial  \Gamma_{\mu\nu}{}^{\lambda}{}_{\rho}}.
\end{equation}
The field equations are:
\begin{align*}
    B_1: & \nabla_\mu \left(\Big[ 2\delta^{[\mu}_\lambda \B{\alpha}{\nu]}{\beta}\B{\gamma}{\rho}{\delta} + 2 \delta^{[\mu}_\lambda \B{\alpha}{\rho]}{\beta} \B{\gamma}{\nu}{\delta} \Big]\deV{\alpha\beta\gamma\delta}\right)\\
    B_2: & \nabla_\mu \left(4 \B{\gamma}{\sigma}{\delta} \B{\sigma}{(\rho}{\lambda} \deV{\nu)\mu\gamma\delta}\right) \\
    B_3: & \nabla_\mu \left(2 \delta^{[\mu}_\lambda \B{\beta}{\nu]}{\gamma} \A{\delta} \deV{\rho\beta\gamma\delta} + 2 \delta^{[\mu}_\lambda \B{\beta}{\rho]}{\gamma} \A{\delta}\deV{\nu\beta\gamma\delta}\right) \\
    B_4: & \nabla_\mu \left(-4 \B{\gamma}{(\rho}{\delta} \A{\lambda} \deV{\nu)\mu\gamma\delta}\right) \\
    B_5: & \nabla_\mu \left(-4 \B{\gamma}{\sigma}{\delta} \A{\sigma} \delta^{(\rho}_\lambda\deV{\nu)\mu\gamma\delta}\right)\\
    C_1: & \nabla_\mu \left(2 \nabla_\beta \B{\gamma}{\rho}{\delta} \delta^{[\mu}_\lambda \deV{\nu]\beta\gamma\delta} + 2 \nabla_\beta \B{\gamma}{\nu}{\delta} \delta^{[\mu}_\lambda \deV{\rho]\beta\gamma\delta}\right) + 2 \Rie{\mu\alpha}{\mu}{\lambda} \B{\gamma}{(\rho}{\delta} \deV{\nu)\alpha\gamma\delta} \\ 
    C_2: & \nabla_\mu \left(-4 \nabla_\sigma \B{\gamma}{\sigma}{\delta} \delta^{(\rho}_\lambda \deV{\nu)\mu\gamma\delta}\right) + 2 \Rie{\alpha\beta}{\sigma}{\sigma} \Big[ 2 \B{\lambda}{(\nu}{\delta} \deV{\rho)\alpha\beta\delta} - \delta^{(\nu}_\lambda \B{\gamma}{\rho)}{\delta} \deV{\alpha\beta\gamma\delta}\Big]\\
    D_1: & 2 \B{\tau}{\sigma}{\lambda} \B{\sigma}{\tau}{\alpha} \B{\gamma}{(\rho}{\delta}\deV{\nu)\alpha\gamma\delta} \\
    D_2: & 2 \B{\alpha}{\sigma}{\beta} \B{\sigma}{(\nu\vert}{\tau} \Big[ 2 \B{\lambda}{\tau}{\delta}\deV{\vert\rho)\alpha\beta\delta} - \delta^{\tau}_\lambda \B{\gamma}{\vert\rho)}{\delta}\deV{\alpha\beta\gamma\delta} \Big] \\
    D_3: & 2 \B{\alpha}{\sigma}{\tau} \B{\beta}{(\nu}{\gamma} \Big[\delta^{\rho)}_\sigma \B{\lambda}{\tau}{\delta} + \delta^{\rho)}_\delta \B{\sigma}{\tau}{\lambda} - \delta^\tau_\lambda \B{\sigma}{|\rho)}{\delta} \Big] \deV{\alpha\beta\gamma\delta} \\
    D_4: & 2\B{\alpha}{\nu}{\beta} \B{\gamma}{\rho}{\delta} \A{\lambda} \deV{\alpha\beta\gamma\delta} \\
    D_5: & 2 \B{\alpha}{(\nu\vert}{\beta} \A{\sigma} \Big[ 2 \B{\lambda}{\sigma}{\delta} \deV{\vert\rho)\alpha\beta\delta} - \delta^\sigma_\lambda \B{\gamma}{\vert\rho)}{\delta} \deV{\alpha\beta\gamma\delta} \Big] \\
    D_6: & -2 \B{\alpha}{(\nu}{\beta} \A{\gamma} \A{\lambda} \deV{\rho)\alpha\beta\gamma} \\
    E_1: & 4 \nabla_\sigma \B{\alpha}{\sigma}{\beta} \Big[ 2 \B{\lambda}{(\rho}{\delta} \deV{\nu)\alpha\beta\delta} - \delta^{(\nu}_\lambda \B{\gamma}{\rho)}{\delta} \deV{\alpha\beta\gamma\delta} \Big] \\
    E_2: & 2 \mathcal{F}_{\alpha\beta} \Big[ 2 \B{\lambda}{(\rho}{\delta} \deV{\nu)\alpha\beta\delta} - \delta^{(\nu}_\lambda \B{\gamma}{\rho)}{\delta} \deV{\alpha\beta\gamma\delta} \Big] 
\end{align*}

\bibliographystyle{unsrt}
\bibliography{References}

\end{document}

